\documentclass[a4paper]{article}
\usepackage[english]{babel}
\usepackage[utf8]{inputenc}
\usepackage[margin=1.15in]{geometry}
\usepackage{amsmath}
\usepackage{setspace}
\usepackage[colorinlistoftodos]{todonotes}
\usepackage{verbatim}
\setlength{\parskip}{\baselineskip}
\setlength{\parindent}{0pt}


\usepackage{enumitem}

\title{QF603 Group Mini-Project 2}

\author{Group F}

\date{\today}

\begin{document}
	\maketitle
	
	\begin{abstract}
		In this report, we implemented a simple linear regression of the Dow Jones Industrial Average (“DJIA”) index over the S\&P500 (“S\&P”) using daily and annual log returns. 
		Our key finding is that the return characteristics of the DJIA and S\&P are not different from each other at the 5\% level of significance. 	
	\end{abstract} 
	
	\newpage
	\setcounter{secnumdepth}{1}
	\section*{Task 3: Regression of Daily Log Returns}
	\label{sec:introduction}
	
	\subsection{Estimate of key statistics $\hat{a},\hat{b}$ and $\hat{\sigma}_{u_t}$}
	\underline{Results}
	\begin{itemize}[nosep]
		\item Alpha = $\hat{a} = -0.000046$, Beta = $\hat{b} = 0.943110$
		\item Standard Distribution of Residual = $\sigma_{u_t} = 0.002877$
	\end{itemize}

	The regression can be expressed as $r_{(DJIA, t)} = -0.000046 +  0.943110 r_{(S\&P500, t)}$.
	
	The positive intercept indicates that the DJIA has a small positive daily excess returns on average as compared to the S\&P. 
	
	The slope of 0.9431 indicates that the DJIA is slightly less volatile than the S\&P. In this context, the slope is interpreted as Beta – the sensitivity of DJIA’s returns to the S\&P’s returns. Assuming investors are risk-averse, a lower Beta is preferred for the same level of return because a lower Beta asset will have more consistency in its returns.
	  
	Combining these two measures together, it suggests that the DJIA has a superior risk-adjusted return as compared to the S\&P. 
	
	
	\subsection{T-test for Null Hypothesis a=b=0 at 5\% significance}
	\underline{Results}
	\begin{itemize}[nosep]
		\item T-statistics for $\hat{a} = 1.486364$, and for $\hat{b} = 339.973655$
		\item Degree of Freedom = $8500 - 2 = 8498$
		\item Null Hypothesis ($H_0$) are that $a=b=0$.
		\item Alternate hypothesis ($H_1$) are that $a\ne 0 \quad \textnormal{and} \quad b\ne 0$
		\item Critical value at 5\% significance level = $\pm1.960243$
	\end{itemize}
	
	\begin{comment}
	For $\hat{a}$, the test statistic lies within the 95\% confidence interval (i.e. does not exceed the 5\% critical level). Therefore, we do not reject the null hypothesis. The intercept coefficient $\hat{a}$ is not significantly different from 0 at the 5\% significance level.
	
	In contrast, for $\hat{b}$, the test statistic lies well beyond the 95\% confidence interval (i.e. significantly exceeds the 5\% critical level). Therefore, we reject the null hypothesis and accept the alternate hypothesis. The slope coefficient $\hat{b}$ is significantly different from 0 at the 5\% significance level.
	\end{comment}
	
	The test statistic for $\hat{a}$ falls within the critical values, and thus we cannot reject the null hypothesis that $\hat{a}=0$. The indication of DJIA having higher daily returns than the S\&P is therefore not significant at the 5\% level.  
	
	However, the test statistic for $\hat{b}$ falls outside the critical values, and thus we reject the null hypothesis that $\hat{b}=0$ and conclude that there is a linear relationship between DJIA and S\&P at the 5\% significance level. 
	
	
	\subsection{Goodness of Fit: $R^2$ and Adjusted $R^2$ values}
	\underline{Results}
	\begin{itemize}[nosep]
		\item R-Squared $(R^2) = 0.931512$
		\item Adjusted R-Squared $(\textnormal{adj-}R^2) = 0.931504$
	\end{itemize}

	The $R^2$ value reports the degree to which our independent variable (the SP500 daily log returns) explains the variation of the dependent variable (the DJIA daily log returns). Since R-square is 0.931512, it means that over 90\% of the variation in the dependent variable is explained by the independent variable.
	
	$\textnormal{Adj-}R^2$ also measures the goodness of model fitting, but takes into consideration the number of independent variables in the model. $R^2$ will only increase or  stay the same when we add more independent variables, even if they do not have any relationship with the dependent variable. $\textnormal{Adj-}R^2$ on the other hand, will "penalize" the model for having excessive dependent variables that do not significantly improve the model. 
	
	The $\textnormal{adj-}R^2$ is always lower than the $R^2$ value. And, in our case, because there is only 1 independent variable, the $\textnormal{adj-}R^2$ and the $R^2$ values are relatively equal.
	
	\subsection{Jarque-Bera test statistic for the residuals}
	\underline{Results}
	\begin{itemize}[nosep]
		\item Jarque-Bera test stats = 25434.27
		\item Degrees of Freedom = lalala
		\item Null Hypothesis $H_0 = 0$
		\item Alternate Hypothesis $H_1 \ne 0$
		\item Critical Chi-Square Value at  ??\% significance level = 5.99146
	\end{itemize}
	
	The JB test statistic exceeds the critical value by a huge margin, strongly indicating that the residuals are not normally distributed. This is due to regression outliers that were a result of extreme market conditions, for example the huge one-day drop on 19-Oct-1987. 
	
	\begin{comment}
	The JB test's null hypothesis is JB = 0. If the null hypothesis is not rejected, it indicates that the
	distribution of the errors are normally distributed (under a certain level of significance).
	\end{comment}
	
	\section*{Task 4: Regression of Yearly Log Returns}
	\label{sec:num2}
	
	\subsection{Estimate of key statistics $\hat{a}$,$\hat{b}$ and $\hat{\sigma_u}$}
	\underline{Results}
	\begin{itemize}[nosep]
		\item Alpha = $\hat{a} = 0.019784$, Beta = $\hat{b} = 0.842545$
		\item Standard Distribution of Residual = $\sigma_{u_t} = 0.037969$
	\end{itemize}

    The regression can be expressed as $r_{(DJIA, t)} = 0.019784 +  0.842545 r_{(S\&P500, t)}$.
	
	The positive intercept indicates that the DJIA has a small positive daily excess returns on average as compared to the S\&P. 
	
	The slope of 0.8425 indicates that the DJIA is slightly less volatile than the S\&P. 
	
	Combining these two measures together, it suggests that the DJIA has a superior risk-adjusted return as compared to the S\&P. 
	
	
	\subsection{T-test for Null Hypothesis a=b=0 at 5\% significance}
	\underline{Results}
	\begin{itemize}[nosep]
		\item T-statistics for $\hat{a} = 2.649890$, and for $\hat{b} = 20.436300$
		\item Degree of Freedom = $32 - 2 = 30$
		\item Null Hypothesis ($H_0$) are that $a=b=0$.
		\item Alternate hypothesis ($H_1$) are that $a\ne 0 \quad \textnormal{and} \quad b\ne 0$
		\item Critical value at 5\% significance level = $\pm2.042272$
	\end{itemize}

    The test statistic for $\hat{a}$ falls outside the critical values, and thus we reject the null hypothesis that $\hat{a}$=0. The t-test at 5\% significance concludes that the DJIA has higher annual returns as compared to the S\&P.  
    
    The test statistic for $\hat{b}$ also falls outside the critical values, and thus we reject the null hypothesis that $\hat{b}$=0 and conclude that there is a linear relationship between the annual returns of the DJIA and the S\&P at the 5\% significance level.

	\subsection{Goodness of Fit: $R^2$ and Adjusted $R^2$ values}
	\underline{Results}
	\begin{itemize}[nosep]
		\item R-Squared $(R^2) = 0.932983$
		\item Adjusted R-Squared $(\textnormal{adj-}R^2) = 0.930749$
	\end{itemize}

     The $R^2$ value is very high, further supporting our previous point that there exists a linear relationship between the annual returns of the DJIA and the S\&P.
	
	\subsection{Jarque-Bera test statistic for the residuals}
	\underline{Results}
	\begin{itemize}[nosep]
		\item Jarque-Bera test stats = 25434.27
		\item Degrees of Freedom = lalala
		\item Null Hypothesis $H_0 = 0$
		\item Alternate Hypothesis $H_1 \ne 0$
		\item Critical Chi-Square Value at  ??\% significance level = 5.99146
	\end{itemize}
	
	The JB test statistic falls within the critical value, indicating that the regression residuals are normally distributed. 
	
\end{document}
