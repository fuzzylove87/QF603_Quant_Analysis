\documentclass[a4paper, 12pt, twoside]{article}
%\usepackage{lipsum}
\usepackage[top=3.5cm, bottom=4cm, left=4cm, right=3.85cm]{geometry}
\usepackage{titlesec}
\usepackage{graphicx}
\usepackage{epstopdf}
\usepackage{float}
\usepackage{amsmath}  %\begin{equation*} ... \end{equation*}
%\renewcommand{\headrulewidth}{0pt}

\newcommand\blfootnote[1]{%
  \begingroup
  \renewcommand{\thefootnote}{}
  \footnote{#1}%
  \addtocounter{footnote}{-1}%
  \endgroup
}
%or
%\makeatletter
%\def\blfootnote{\xdef\@thefnmark{}\@footnotetext}
%\makeatother


\renewcommand{\figurename}{F\textsc{ig}.}
\setcounter{page}{637}
\titleformat*{\section}{\bfseries}

\renewenvironment{abstract}
 {\linespread{1.15}\fontsize{10}{10}\selectfont
  %\begin{center}
  %\bfseries \abstractname\vspace{-.5em}\vspace{0pt}
  %\end{center}
  \list{}{
    \setlength{\leftmargin}{0.9cm}%
    \setlength{\rightmargin}{\leftmargin}%
  }%
  \item\relax}
 {\endlist}

\usepackage{fancyhdr}
\pagestyle{fancy} 
\fancyhead{}
\fancyhead[LE]{{\fontsize{11}{11}\selectfont \thepage}}
\fancyhead[RO]{{\fontsize{11}{11}\selectfont \thepage}}
\fancyhead[RE]{{\fontsize{11}{11}\selectfont JOURNAL OF POLITICAL ECONOMY}}
\fancyhead[LO]{{\fontsize{11}{11}\selectfont OPTIONS AND LIABILITIES}}
\fancyfoot[L,R,C]{}
\renewcommand{\headrulewidth}{0pt}
\renewcommand*\footnoterule{}
\fancypagestyle{plain}{
   	\fancyhead{}
  	\fancyfoot{}
    \fancyfoot[C]{\thepage}
}
\begin{document}
\setlength{\abovedisplayskip}{1pt}  %before equation space
\setlength{\belowdisplayskip}{1pt}  %after equation space
\thispagestyle{plain}
\begingroup 
\linespread{1.5}\fontfamily{phv}\fontsize{18}{18}\bfseries\selectfont
\noindent The Pricing of Options and Corporate \\
Liabilities \par
\endgroup 

\vspace{20mm}

\hrule 

\vspace{5mm}

\begingroup 
\linespread{1.5}\fontfamily{phv}\fontsize{14}{14}\bfseries\selectfont
\noindent Fischer Black \blfootnote{\linespread{1}\fontsize{10}{10}\selectfont Received for publication November 11, 1970. Final version received May 9, 1972.} \blfootnote{\linespread{1}\fontsize{10}{10}\selectfont The inspiration for this work was provided by Jack L. Treynor (1961a, 1961b). We are grateful for extensive comments on earlier drafts by Eugene F. Fama, Robert C. Merton, and Merton H. Miller. This work was supported in part by the Ford Foundation.} \par
\endgroup 

\vspace{1mm}

\begingroup 
\linespread{1.5}\fontsize{8}{8}\selectfont
\noindent \textit{University of Chicago} \par
\endgroup 

\vspace{2mm}

\begingroup 
\linespread{1.5}\fontfamily{phv}\fontsize{14}{14}\bfseries\selectfont
\noindent Myron Scholes \par
\endgroup 

\vspace{1mm}

\begingroup 
\linespread{1.5}\fontsize{8}{8}\selectfont
\noindent \textit{Massachusetts Institute of Technology} \par
\endgroup 

\vspace{8mm}

\begin{abstract}

\end{abstract}

\vspace{2mm}

\linespread{1.15}\selectfont 
\section*{Introduction}

\newgeometry{top=3.3cm, bottom=3.8cm, left=4cm, right=3.85cm}



$$
k_x N(b_1)- k^* c N(b_2)
$$

$$
b_1 = \frac{\ln kx/c + {\displaystyle \frac{1}{2}} v^2 (t^*-t)}{v \sqrt{}(t^*-t)}
$$

$$
b_2 = \frac{\ln kx/c - {\displaystyle \frac{1}{2}} v^2 (t^*-t)}{v \sqrt{}(t^*-t)}
$$

\vspace*{3mm}

\noindent In this expression, $x$ is the stock price, $c$ is the exercise price, $t^*$ is the maturity 
date, $t$ is the current date, $v^2$ is the variance rate of the return on the stock, $\ln$
is the natural logarithm, and $N(b)$ is the cumulative normal density function. But $k$ 
and $k^*$ are unknown parameters. Sprenkle (1961) defines $k$ as the ratio of the expected 
value of the stock price at the time the warrant matures to the current stock price, 
and $k^*$ as a discount factor that depends on the risk of the stock. He tries to 
estimate the values of $k$ and $k^*$ empirically, but finds that he is unable to do so.

More typically, Samuelson (1965) has unknown parameters $\alpha$ and $\beta$, where $\alpha$ is the rate 
of expected return on the stock, \footnote{\linespread{1}\fontsize{10}{10}\selectfont 
The variance rate of the return on a security is the limit, as the size 
of the interval of measurement goes to zero, of the variance of the return over that 
interval divided by the length of the interval.} 
and $\beta$ is the rate of expected return on the warrant 
or the discount rate to be applied to the warrant. \footnote{\linespread{1}\fontsize{10}{10}\selectfont The rate of expected return on a security is the limit, as the size of 
the interval of measurement goes to zero, of the expected return over that interval 
divided by the length of the interval.}  
He assumes that the distribution of 
possible values of the stock when the warrant matures is log-normal and takes the 
expected value of this distribution, cutting it off at the exercise price. He then 
discounts this expected value to the present at the rate $\beta$. Unfortunately, there seems 
to be no model of the pricing of securities under conditions of capital market 
equilibrium that would make this an appropriate procedure for determining the value of 
a warrant.

In a subsequent paper, Samuelson and Merton (1969) recognize the fact that discounting 
the expected value of the distribution of possible values of the warrant when it is 
exercised is not an appropriate procedure. They advance the theory by treating the 
option price as a function of the stock price. They also recognize that the discount 
rates are determined in part by the requirement that investors be willing to hold all 
of the outstanding amounts of both the stock and the option. But they do not make use 
of the fact that investors must hold other assets as well, so that the risk of an 
option or stock that affects its discount rate is only that part of the risk that 
cannot be diversified away. Their final formula depends on the shape of the utility 
function that they assume for the typical investor.

One of the concepts that we use in developing our model is expressed by Thorp and 
Kassouf (1967). They obtain an empirical valuation formula for warrants by fitting a 
curve to actual warrant prices. Then they use this formula to calculate the ratio of 
shares of stock to options needed to create a hedged position by going long in one 
security and short in the other. What they fail to pursue is the fact that in 
equilibrium, the expected return on such a hedged position must be equal to the return 
on a riskless asset. What we show below is that this equilibrium condition can be used 
to derive a theoretical valuation formula.

\section*{The Valuation Formula}

In deriving our formula for the value of an option in terms of the price of the stock, 
we will assume ``ideal conditions" in the market for the stock and for the option:

\textit{a}) The short-term interest rate is known and is constant through time.

\textit{b}) The stock price follows a random walk in continuous time with a variance rate 
proportional to the square of the stock price. Thus the distribution of possible stock 
prices at the end of any finite interval is lognormal. The variance rate of the return 
on the stock is constant.

\textit{c}) The stock pays no dividends or other distributions.

\textit{d}) The option is ``European," that is, it can only be exercised at maturity.

\textit{e}) There are no transaction costs in buying or selling the stock or the option.

\textit{f}) It is possible to borrow any fraction of the price of a security to buy it or to 
hold it, at the short-term interest rate.

\textit{g}) There are no penalties to short selling. A seller who does not own a security will 
simply accept the price of the security from a buyer, and will agree to settle with 
the buyer on some future date by paying him an amount equal to the price of the 
security on that date.

Under these assumptions, the value of the option will depend only on the price of the 
stock and time and on variables that are taken to be known constants. Thus, it is 
possible to create a hedged position, consisting of a long position in the stock and a 
short position in the option, whose value will not depend on the price of the stock, 
but will depend only on time and the values of known constants. Writing $w(x,t)$ for the 
value of the option as a function of the stock price $x$ and time $t$, the number of 
options that must be sold short against one share of stock long is:
\begin{equation} 
1/w_1(x,t) %\label{eqn01}
\end{equation}

\vspace*{1mm}
\begin{equation}
x-w/w_1.
\end{equation}

\vspace*{1mm}
\noindent The change in the value of the equity in a short interval $\Delta t$ is:
\vspace*{1mm}
\begin{equation}
\Delta x-\Delta w/w_1.
\end{equation}
\noindent Assuming that the short position is changed continuously, we can use stochastic 
calculus \footnote{\linespread{1}\fontsize{10}{10}\selectfont
For an exposition of stochastic calculus, see McKean (1969).
}
to expand $\Delta w$, which is $w(x+\Delta x,t+\Delta t)-w(x,t)$, as follows:
\vspace*{1mm}
\begin{equation}
\Delta w=w_1 \Delta x+\frac{1}{2} w_{11} v^2 x^2 \Delta t+w_2 \Delta t
\end{equation}

\vspace*{1mm}
\noindent In equation (4), the subscripts on $w$ refer to partial derivatives, and $v^2$ is the 
variance rate of the return on the stock. \footnote{\linespread{1}\fontsize{10}{10}\selectfont
See footnote 1.
}
Substituting from equation (4) in to 
expression (3), we find that the change in the value of the equity in the hedged 
position is:
\vspace*{1mm}
\begin{equation}
-\left({\displaystyle \frac{1}{2}} w_11 v^2 x^2+w_2 \right)\Delta t/w_1
\end{equation}

\vspace*{1mm}
Since the return on the equity in the hedged position is certain, the return must be 
equal to $r\Delta t$. Even if the hedged position is not changed continuously, its risk is 
small and is entirely risk that can be diversified away, so the expected return on the 
hedged position must be at the short term interest rate. 
\footnote{\linespread{1}\fontsize{10}{10}\selectfont
For a thorough discussion of the relation between risk and expected 
return, see Fama and Miller (1972) or Sharpe (1970). To see that the risk in the 
hedged position can be diversified away, note that if we don't adjust the hedge 
continuously, expression (5) becomes:
\vspace*{1mm}
\begin{equation*}
-\left(\frac{1}{2} w_{11} \Delta x^2+w_2 \Delta t\right)/w_1 \tag{5'}
\end{equation*}

\vspace*{1mm}
\noindent Writing $\Delta m$ for the change in the value of the market portfolio 
between $t$ and $t+\Delta t$, the 
``market risk" in the hedged position is proportional to the covariance between the 
change in the value of the hedged portfolio, as given by expression (5'), 
and $\Delta m: -\frac{1}{2} w_{11} cov(\Delta x^2,\Delta m)$. 
But if $\Delta x$ and $\Delta m$ follow a joint normal distribution for small 
intervals $\Delta t$, this covariance will be zero. 
Since there is no market risk in the 
hedged position, all of the risk due to the fact that the hedge is not continuously 
adjusted must be risk that can be diversified away.
}
If this were not true, speculators would try to profit by borrowing large amounts of money to create such 
hedged positions, and would in the process force the returns down to the short term 
interest rate.

Thus the change in the equity (5) must equal the value of the equity (2) times $r\Delta t$.

\begin{equation}
-\left(\frac{1}{2} w_{11} v^2 x^2+w_2 \right)\Delta t/w_1 =(x-w/w_1) r\Delta t.
\end{equation}

\vspace*{1mm}
\noindent Dropping the $\Delta t$ from both sides, and rearranging, we have a differential equation for 
the value of the option.

\begin{equation}
w_2=rw-rxw_1- \frac{1}{2} v^2 x^2 w_{11}.
\end{equation}

\vspace*{1mm}
\noindent Writing $t^*$ for the maturity date of the option, and $c$ for the exercise price, we know 
that:

\begin{equation}
\begin{array}{rll}
w(x,t^*) &=x-c, &x \ge c \\
         &=0,   &x < c
\end{array} 
\end{equation}

\vspace*{1mm}
\noindent There is only one formula $w(x,t)$ that satisfies the differential equation (7) subject 
to the boundary condition (8). This formula must be the option valuation formula.
To solve this differential equation, we make the following substitution:

\setlength{\jot}{5pt}
\begin{multline}
w(x,t)= e^{r(t-t^*)} y \left[ (2/v^2) \left( r-\frac{1}{2} v^2 \right) \right.\\
\left[\ln x/c - \left( r-\frac{1}{2}v^2 \right)(t-t^*) \right] \\
\left.-(2/v^2) \left(r-\frac{1}{2}v^2 \right)^2 \right].
\end{multline}

With this substitution, the differential equation becomes:

\begin{equation}
y_2=y_{11},
\end{equation}
and the boundary condition becomes:

\begin{equation}
\begin{array}{rll}
y(u,0) &=0, & u < 0 \\
       &=c \left[e^{u\left( \frac{1}{2} v^2\right)\big/ \left(r- \frac{1}{2} v^2\right)}-1 \right], & u \ge 0
\end{array} 
\end{equation}

The differential equation (10) is the heat-transfer equation of physics, and its 
solution is given by Churchill (1963, p. 155). In our notation, the solution is:

\begin{multline}
y(u,s)=1/\sqrt{2\pi} \int_{-u/\sqrt{2s}}^{\infty}\\
c\left[ e^{(u+q\sqrt{2s})\left( \frac{1}{2} v^2\right)\big/ \left(r- \frac{1}{2} v^2\right)}-1 \right]e^{-q^2/2}dq.
\end{multline}

\vspace*{3mm}
Substituting from equation (12) into equation (9), and simplifying, we find:

\begin{eqnarray}
w(x,t) &=& xN(d_1)-ce^{r(t-t^*)} N(d_2) \nonumber \\
d_1&=& \frac{\ln x/c+(r+\displaystyle \frac{1}{2} v^2)(t^*-t)}{v\sqrt{t^*-t})} \\
d_2&=& \frac{\ln x/c+(r-\displaystyle \frac{1}{2} v^2)(t^*-t)}{v\sqrt{t^*-t})} \nonumber
\end{eqnarray}

\vspace*{3mm}
\noindent In equation (13), $N(d)$ is the cumulative normal density function.

Note that the expected return on the stock does not appear in equation (13). The 
option value as a function of the stock price is independent of the expected return on 
the stock. The expected return on the option, however, will depend on the expected 
return on the stock. The faster the stock price rises, the faster the option price 
will rise through the functional relationship (13).

Note that the maturity $(t^*-t)$ appears in the formula only multiplied by the interest 
rate r or the variance rate $v^2$. Thus, an increase in maturity has the same effect on 
the value of the option as an equal percentage increase in both $r$ and $v^2$.

Merton (1973) has shown that the option value as given by equation (13) increases 
continuously as any one of $t^*$, $r$, or $v^2$ increases. In each case, it approaches a 
maximum value equal to the stock price.

The partial derivative $w_1$ of the valuation formula is of interest, because it 
determines the ratio of shares of stock to options in the hedged position as in 
expression (1). Taking the partial derivative of equation (13), and simplifying, we 
find that:

\begin{equation}
w_1(x,t)=N(d_1).
\end{equation}

\vspace*{1mm}
\noindent In equation (14), $d_1$ is as defined in equation (13).

From equations (13) and (14), it is clear that $xw_1/w$ is always greater than one. This 
shows that the option is always more volatile than the stock.

\section*{An Alternative Derivation}

It is also possible to derive the differential equation (7) using the ``capital asset 
pricing model." This derivation is given because it gives more understanding of the 
way in which one can discount the value of an option to the present, using a discount 
rate that depends on both time and the price of the stock.

The capital asset pricing model describes the relation between risk and expected 
return for a capital asset under conditions of market equilibrium. 
\footnote{\linespread{1}\fontsize{10}{10}\selectfont
The model was developed by Treynor (1961b); Sharpe (1964), Lintner 
(1965), and Mossin (1966). It is summarized by Sharpe (1970), and Fama and Miller 
(1972). The model was originally stated as a single-period model. Extending it to a 
multi-period model is, in general, difficult. Fama (1970), however, has shown that if 
we make an assumption that implies that the short-term interest rate is constant 
through time, then the model must apply to each successive period in time. His proof 
also goes through under somewhat more general assumptions.
}
The expected return 
on an asset gives the discount that must be applied to the end-of-period value of the 
asset to give its present value. Thus, the capital-asset pricing model gives a general 
method for discounting under uncertainty.

The capital-asset pricing model says that the expected return on an asset is a linear 
function of its $\beta$, which is defined as the covariance of the return on the asset with 
the return on the market, divided by the variance of the return on the market. From 
equation (4) we see that the covariance of the return on the option $\Delta w/w$ with the 
return on the market is equal to $xw_1/w$ times the covariance of the return on the 
stock $\Delta x/x$ with the return on the market. Thus, we have the following relation between 
the option's $\beta$ and the stock's $\beta$:

\begin{equation}
\beta_w=(xw_1/w)\beta_x.
\end{equation}

\noindent The expression $xw_1/w$ may also be interpreted as the ``elasticity" of the option price 
with respect to the stock price. It is the ratio of the percentage change in the 
option price to the percentage change in the stock price, for small percentage 
changes, holding maturity constant.

To apply the capital-asset pricing model to an option and the underlying stock, let us 
first define a as the rate of expected return on the market minus the interest rate.
\footnote{\linespread{1}\fontsize{10}{10}\selectfont
See footnote 2.
}
Then the expected return on the option and the stock are:

\begin{eqnarray}
E(\Delta x/x)&=&r\Delta t+a\beta_x \Delta t, \\
E(\Delta w/w)&=&r\Delta t+a\beta_w \Delta t. 
\end{eqnarray}

\vspace*{3mm}
\noindent Multiplying equation (17) by w, and substituting for $\beta_w$ from equation (15), we find:

\begin{equation}
E(\Delta w)=rw\Delta t+axw_1 \beta_x \Delta t.      
\end{equation}

Using stochastic calculus,\footnote{\linespread{1}\fontsize{10}{10}\selectfont
For an exposition of stochastic calculus, see McKean (1969).
}
we can expand $\Delta w$, which is $w(x+\Delta x,t+\Delta t)-w(x,t)$, as follows:

\begin{equation}
\Delta w=w_1 \Delta x+ \frac{1}{2} w_{11} v^2 x^2 \Delta t+w_2 \Delta t.
\end{equation}

\noindent Taking the expected value of equation (19), and substituting for $E(\Delta x)$ from equation 
(16), we have:

\begin{equation}
E(\Delta w)=rxw_1 \Delta t+axw_1 \beta_x \Delta t+ \frac{1}{2} v^2 x^2 w_{11} \Delta t+w_2 \Delta t.
\end{equation}

\noindent Combining equations (18) and (20), we find that the terms involving a and $\beta_x$ cancel, 
giving:

\begin{equation}
w_2=rw-rxw_1-\frac{1}{2} v^2 x^2 w_{11}. 
\end{equation}

\noindent Equation (21) is the same as equation (7).

\section*{More Complicated Options}

The valuation formula (13) was derived under the assumption that the option can only 
be exercised at time $t^*$. Merton (1973) has shown, however, that the value of the 
option is always greater than the value it would have if it were exercised immediately 
$(x-c)$.  Thus, a rational investor will not exercise a call option before maturity, and 
the value of an American call option is the same as the value of a European call    
option.

There is a simple modification of the formula that will make it applicable to European 
put options (options to sell) as well as call options (options to buy). Writing $u(x,t)$ 
for the value of a put option, we see that the differential equation remains 
unchanged.

\begin{equation}
u_2=ru-rxu_1- \frac{1}{2} v^2 x^2 u_{11}.
\end{equation}

\vspace*{3mm}
The boundary condition, however, becomes:

\begin{equation}
\begin{array}{rll}
u(x,t^*) &=0, &x \ge c \\
         &=c-x,   &x < c
\end{array} 
\end{equation}

To get the solution to this equation with the new boundary condition, we can simply 
note that the difference between the value of a call and the value of a put on the 
same stock, if both can be exercised only at maturity, must obey the same differential 
equation, but with the following boundary condition:

\begin{equation}
w(x,t^*)-u(x,t^*)=x-c.
\end{equation}

The solution to the differential equation with this boundary condition is:

\begin{equation}
w(x,t)-u(x,t)=x-ce^r(t-t^*).
\end{equation}

Putting in the value of $w(x,t)$ from equation (13), and noting that $1-N(d)$ is equal to 
$N(-d)$, we have:

\begin{equation}
u(x,t)=-xN(-d_1)+ce^(-rt^*) N(-d_2).
\end{equation}

\noindent In equation (27), $d_1$ and $d_2$ are defined as in equation (13).

Equation (25) also gives us a relation between the value of a European call and the 
value of a European put. \footnote{\linespread{1}\fontsize{10}{10}\selectfont
The relation between the value of a call option and the value of a put 
option was first noted by Stoll (1969). He does not realize, however, that his 
analysis applies only to European options.
}
We see that if an investor were to buy a call and sell a put, 
his returns would be exactly the same as if he bought the stock on margin, borrowing 
$ce^r(t-t^*)$ toward the price of the stock.

Merton (1973) has also shown that the value of an American put option will be greater 
than the value of a European put option. This is true because it is sometimes 
advantageous to exercise a put option before maturity, if it is possible to do so. For 
example, suppose the stock price falls almost to zero and that the probability that 
the price will exceed the exercise price before the option expires is negligible. Then 
it will pay to exercise the option immediately, so that the exercise price will be 
received sooner rather than later. The investor thus gains the interest on the 
exercise price for the period up to the time he would otherwise have exercised it. So 
far, no one has been able to obtain a formula for the value of an American put option.

If we relax the assumption that the stock pays no dividend, we begin to get into some 
complicated problems. First of all, under certain conditions it will pay to exercise 
an American call option before maturity. Merton (1973) has shown that this can be true 
only just before the stock's ex-dividend date. Also, it is not clear what adjustment 
might be made in the terms of the option to protect the option holder against a loss 
due to a large dividend on the stock and to ensure that the value of the option will 
be the same as if the stock paid no dividend. Currently, the exercise price of a call 
option is generally reduced by the amount of any dividend paid on the stock. We can 
see that this is not adequate protection by imagining that the stock is that of a 
holding company and that it pays out all of its assets in the form of a dividend to 
its shareholders. This will reduce the price of the stock and the value of the option 
to zero, no matter what adjustment is made in the exercise price of the option. In 
fact, this example shows that there may not be any adjustment in the terms of the 
option that will give adequate protection against a large dividend. In this case, the 
option value is going to be zero after the distribution, no matter what its terms are. 
Merton (1973) was the first to point out that the current adjustment for dividends is 
not adequate.

\section*{Warrant Valuation}

A warrant is an option that is a liability of a corporation. The holder of a warrant 
has the right to buy the corporation's stock (or other assets) on specified terms. The 
analysis of warrants is often much more complicated than the analysis of simple 
options, because:

\textit{a}) The life of a warrant is typically measured in years, rather than months. Over a 
period of years, the variance rate of the return on the stock may be expected to 
change substantially.

\textit{b}) The exercise price of the warrant is usually not adjusted at all for dividends. The 
possibility that dividends will be paid requires a modification of the valuation 
formula.

\textit{c}) The exercise price of a warrant sometimes changes on specified dates. It may pay to 
exercise a warrant just before its exercise price changes. This too requires a 
modification of the valuation formula.

\textit{d}) If the company is involved in a merger, the adjustment that is made in the terms of 
the warrant may change its value.

\textit{e}) Sometimes the exercise price can be paid using bonds of the corporation at face 
value, even though they may at the time be selling at a discount. This complicates the 
analysis and means that early exercise may sometimes be desirable.

\textit{f}) The exercise of a large number of warrants may sometimes result in a significant 
increase in the number of common shares outstanding.

In some cases, these complications can be treated as insignificant, and equation (13) 
can be used as an approximation to give an estimate of the warrant value. In other 
cases, some simple modifications of equation (13) will improve the approximation. 
Suppose, for example, that there are warrants outstanding, which, if exercised, would 
double the number of shares of the company's common stock. Let us define the ``equity" 
of the company as the sum of the value of all of its warrants and the value of all of 
its common stock. If the warrants are exercised at maturity, the equity of the company 
will increase by the aggregate amount of money paid in by the warrant holders when 
they exercise. The warrant holders will then own half of the new equity of the 
company, which is equal to the old equity plus the exercise money.

Thus, at maturity, the warrant holders will either receive nothing, or half of the new 
equity, minus the exercise money. Thus, they will receive nothing or half of the 
difference between the old equity and half the exercise money.  We can look at the 
warrants as options to buy shares in the equity rather than shares of common stock, at 
half the stated exercise price rather than at the full exercise price. The value of a 
share in the equity is defined as the sum of the value of the warrants and the value 
of the common stock, divided by twice the number of outstanding shares of common 
stock. If we take this point of view, then we will take $v^2$ in equation (13) to be the 
variance rate of the return on the company's equity, rather than the variance rate of 
the return on the company's common stock.

A similar modification in the parameters of equation (13) can be made if the number of 
shares of stock outstanding after exercise of the warrants will be other than twice 
the number of shares outstanding before exercise of the warrants.

\section*{Common Stock and Bond Valuation}

It is not generally realized that corporate liabilities other than warrants may be 
viewed as options. Consider, for example, a company that has common stock and bonds 
outstanding and whose only asset is shares of common stock of a second company. 
Suppose that the bonds are ``pure discount bonds" with no coupon, giving the holder the 
right to a fixed sum of money, if the corporation can pay it, with a maturity of 10 
years. Suppose that the bonds contain no restrictions on the company except a 
restriction that the company cannot pay any dividends until after the bonds are paid 
off. Finally, suppose that the company plans to sell all the stock it holds at the end 
of 10 years, pay off the bond holders if possible, and pay any remaining money to the 
stockholders as a liquidating dividend.

Under these conditions, it is clear that the stockholders have the equivalent of an 
option on their company's assets. In effect, the bond holders own the company's 
assets, but they have given options to the stockholders to buy the assets back. The 
value of the common stock at the end of 10 years will be the value of the company's 
assets minus the face value of the bonds, or zero, whichever is greater.

Thus, the value of the common stock will be $w(x,t)$, as given by equation (13), where 
we take $v^2$ to be the variance rate of the return on the shares held by the company, $c$ 
to be the total face value of the outstanding bonds, and $x$ to be the total value of 
the shares held by the company. The value of the bonds will simply be $x-w(x,t)$.

By subtracting the value of the bonds given by this formula from the value they would 
have if there were no default risk, we can figure the discount that should be applied 
to the bonds due to the existence of default risk.

Suppose, more generally, that the corporation holds business assets rather than 
financial assets. Suppose that at the end of the 10 year period, it will recapitalize 
by selling an entirely new class of common stock, using the proceeds to pay off the 
bond holders, and paying any money that is left to the old stockholders to retire 
their stock. In the absence of taxes, it is clear that the value of the corporation 
can be taken to be the sum of the total value of the debt and the total value of the 
common stock.\footnote{\linespread{1}\fontsize{10}{10}\selectfont  
The fact that the total value of a corporation is not affected by its 
capital structure, in the absence of taxes and other imperfections, was first shown by 
Modigliani and Miller (1985).
}
The amount of debt outstanding will not affect the total value of the 
corporation, but will affect the division of that value between the bonds and the 
stock. The formula for $w(x,t)$ will again describe the total value of the common stock, 
where $x$ is taken to be the sum of the value of the bonds and the value of the stock. 
The formula for $x-w(x,t)$ will again describe the total value of the bonds. It can be 
shown that, as the face value c of the bonds increases, the market value $x-w(x,t)$ 
increases by a smaller percentage. An increase in the corporation's debt, keeping the 
total value of the corporation constant, will increase the probability of default and 
will thus reduce the market value of one of the corporation's bonds. If the company 
changes its capital structure by issuing more bonds and using the proceeds to retire 
common stock, it will hurt the existing bond holders, and help the existing 
stockholders. The bond price will fall, and the stock price will rise. In this sense, 
changes in the capital structure of a firm may affect the price of its common stock.
\footnote{\linespread{1}\fontsize{10}{10}\selectfont  
For a discussion of this point, see Fama and Miller (1972, pp. 151-52).
}  
The price changes will occur when the change in the capital structure becomes certain, 
not when the actual change takes place.

Because of this possibility, the bond indenture may prohibit the sale of additional 
debt of the same or higher priority in the event that the firm is recapitalized. If 
the corporation issues new bonds that are subordinated to the existing bonds and uses 
the proceeds to retire common stock, the price of the existing bonds and the common 
stock price will be unaffected. Similarly, if the company issues new common stock and 
uses the proceeds to retire completely the most junior outstanding issue of bonds, 
neither the common stock price nor the price of any other issue of bonds will be 
affected.

The corporation's dividend policy will also affect the division of its total value 
between the bonds and the stock.\footnote{\linespread{1.15}\fontsize{10}{10}\selectfont  
Miller and Modigliani (1961) show that the total value of a firm, in the 
absence of taxes and other imperfections, is not affected by its dividend policy. They 
also note that the price of the common stock and the value of the bonds will not be 
affected by a change in dividend policy if the funds for a higher dividend are raised 
by issuing common stock or if the money released by a lower dividend is used to 
repurchase common stock.} To take an extreme example, suppose again that the 
corporation's only assets are the shares of another company, and suppose that it sells 
all these shares and uses the proceeds to pay a dividend to its common stockholders. 
Then the value of the firm will go to zero, and the value of the bonds will go to 
zero. The common stockholders will have ``stolen" the company out from under the bond 
holders. Even for dividends of modest size, a higher dividend always favors the 
stockholders at the expense of the bond holders. A liberalization of dividend policy 
will increase the common stock price and decrease the bond price. \footnote{\linespread{1.15}\fontsize{10}{10}\selectfont  
This is true assuming that the liberalization of dividend policy is not 
accompanied by a change in the company's current and planned financial structure. 
Since the issue of common stock or junior debt will hurt the common shareholders 
(holding dividend policy constant), they will normally try to liberalize dividend 
policy without issuing new securities. They may be able to do this by selling some of 
the firm's financial assets, such as ownership claims on other firms. Or they may be 
able to do it by adding to the company's short-term bank debt, which is normally 
senior to its long-term debt. Finally, the company may be able to finance a higher 
dividend by selling off a division. Assuming that it receives a fair price for the 
division, and that there were no economies of combination, this need not involve any 
loss to the firm as a whole. If the firm issues new common stock or junior debt in 
exactly the amounts needed to finance the liberalization of dividend policy, then the 
common stock and bond prices will not be affected. If the liberalization of dividend 
policy is associated with a decision to issue more common stock or junior debt than is 
needed to pay the higher dividends, the common stock price will fall and the bond 
price will rise. But these actions are unlikely, since they are not in the 
stockholders' best interests.} Because of this 
possibility, bond indentures contain restrictions on dividend policy, and the common 
stockholders have an incentive to pay themselves the largest dividend allowed by the 
terms of the bond indenture. However, it should be noted that the size of the effect 
of changing dividend policy will normally be very small.

If the company has coupon bonds rather than pure discount bonds outstanding, then we 
can view the common stock as a ``compound option." The common stock is an option on an 
option on ... an option on the firm. After making the last interest payment, the 
stockholders have an option to buy the company from the bond holders for the face 
value of the bonds. Call this ``option 1." After making the next-to-the-last interest 
payment, but before making the last interest payment, the stockholders have an option 
to buy option 1 by making the last interest payment. Call this ``option 2." Before 
making the next-to-the-last interest payment, the stockholders have an option to buy 
option 2 by making that interest payment. This is ``option 3." The value of the 
stockholders' claim at any point in time is equal to the value of option $n+1$, where $n$ 
is the number of interest payments remaining in the life of the bond.

If payments to a sinking fund are required along with interest payments, then a 
similar analysis can be made. In this case, there is no ``balloon payment" at the end 
of the life of the bond. The sinking fund will have a final value equal to the face 
value of the bond. Option 1 gives the stockholders the right to buy the company from 
the bond holders by making the last sinking fund and interest payment. Option 2 gives 
the stockholders the right to buy option 1 by making the next-to-the-last sinking fund 
and interest payment. And the value of the stockholders' claim at any point in time is 
equal to the value of option $n$, where $n$ is the number of sinking fund and interest 
payments remaining in the life of the bond. It is clear that the value of a bond for 
which sinking fund payments are required is greater than the value of a bond for which 
they are not required.

If the company has callable bonds, then the stockholders have more than one option. 
They can buy the next option by making the next interest or sinking fund and interest 
payment, or they can exercise their option to retire the bonds before maturity at 
prices specified by the terms of the call feature. Under our assumption of a constant 
short-term interest rate, the bonds would never sell above face value, and the usual 
kind of call option would never be exercised. Under more general assumptions, however, 
the call feature would have value to the stockholders and would have to be taken into 
account in deciding how the value of the company is divided between the stockholders 
and the bond holders.

Similarly, if the bonds are convertible, we simply add another option to the package. 
It is an option that the bond holders have to buy part of the company from the 
stockholders.

Unfortunately, these more complicated options cannot be handled by using the valuation 
formula (13). The valuation formula assumes that the variance rate of the return on 
the optioned asset is constant. But the variance of the return on an option is 
certainly not constant: it depends on the price of the stock and the maturity of the 
option. Thus the formula cannot be used, even as an approximation, to give the value 
of an option on an option. It is possible, however, that an analysis in the same 
spirit as the one that led to equation (13) would allow at least a numerical solution 
to the valuation of certain more complicated options.

\section*{Empirical Tests}

We have done empirical tests of the valuation formula on a large body of call-option 
data (Black and Scholes 1972). These tests indicate that the actual prices at which 
options are bought and sold deviate in certain systematic ways from the values 
predicted by the formula. Option buyers pay prices that are consistently higher than 
those predicted by the formula. Option writers, however, receive prices that are at 
about the level predicted by the formula. There are large transaction costs in the 
option market, all of which are effectively paid by option buyers.

Also, the difference between the price paid by option buyers and the value given by 
the formula is greater for options on low-risk stocks than for options on high-risk 
stocks. The market appears to underestimate the effect of differences in variance rate 
on the value of an option. Given the magnitude of the transaction costs in this 
market, however, this systematic misestimation of value does not imply profit 
opportunities for a speculator in the option market.

\section*{References}

\begingroup
\linespread{1.15}\fontsize{10}{10}\selectfont

\hangindent 5mm
Ayres, Herbert F. ``Risk Aversion in the Warrants Market." \textit{Indus. Management Rev.} 4 
(Fall 1963): 497-505.  Reprinted in Cootner (1967), pp. 497-505.

\hangindent 5mm
\noindent Baumol, William J.; Malkiel, Burton G.; and Quandt, Richard E. ``The Valuation of 
Convertible Securities." \textit{Q.J.E.} 80 (February 1966): 48-59.

\hangindent 5mm
\noindent Black, Fischer, and Scholes, Myron. ``The Valuation of Option Contracts and a Test of 
Market Efficiency." \textit{J. Finance} 27 (May 1972): 399-417.

\hangindent 5mm
\noindent Boness, A. James. ``Elements of a Theory of Stock-Option Values." \textit{J.P.E.} 72 (April 
1964): 163-75.

\hangindent 5mm
\noindent Chen, Andrew H. Y. ``A Model of Warrant Pricing in a Dynamic Market." \textit{J. Finance} 25 
(December 1970): 1041-60.

\hangindent 5mm
\noindent Churchill, R. V. \textit{Fourier Series and Boundary Value Problems}, 2d ed. New York:   
McGraw-Hill, 1963.

\hangindent 5mm
\noindent Cootner, Paul A. \textit{The Random Character of Stock Market Prices}. Cambridge, Mass.: M.I.T. 
Press, 1967.

\hangindent 5mm
\noindent Fama, Eugene F. ``Multiperiod Consumption-Investment Decisions." \textit{A.E.R.} 60 (March 
1970): 163-74.

\hangindent 5mm
\noindent Fama, Eugene F., and Miller, Merton H. \textit{The Theory of Finance}. New York: Holt, Rinehart 
\& Winston, 1972.

\hangindent 5mm
\noindent Lintner, John. ``The Valuation of Risk Assets and the Selection of Risky Investments in 
Stock Portfolios and Capital Budgets." \textit{Rev. Econ. and Statis.} 47 (February 1965): 768-83.

\hangindent 5mm
\noindent McKean, H. P., Jr. \textit{Stochastic Integrals}. New York: Academic Press, 1969.

\hangindent 5mm
\noindent Merton, Robert C. ``Theory of Rational Option Pricing." \textit{Bell J. Econ. and Management  
Sci.} (1973): in press.

\hangindent 5mm
\noindent Miller, Merton H., and Modigliani, Franco. ``Dividend Policy, Growth, and the Valuation 
of Shares." \textit{J. Bus.} 34 (October 1961): 411-33.

\hangindent 5mm
\noindent Modigliani, Franco, and Miller, Merton H. ``The Cost of Capital, Corporation Finance, 
and the Theory of Investment." \textit{A.E.R.} 48 (June 1958): 261-97. 

\hangindent 5mm
\noindent Mossin, Jan. ``Equilibrium in a Capital Asset Market." \textit{Econometrica} 34 (October 1966): 
768-83.

\hangindent 5mm
\noindent Samuelson, Paul A. ``Rational Theory of Warrant Pricing." \textit{Indus. Management Rev.} 6 
(Spring 1965): 13-31. Reprinted in Cootner (1967), pp. 506-32.

\hangindent 5mm
\noindent Samuelson, Paul A., and Merton, Robert C. ``A Complete Model of Warrant Pricing that 
Maximizes Utility." \textit{Indus. Management Rev.} 10 (Winter 1969): 17-46.

\hangindent 5mm
\noindent Sharpe, William F. ``Capital Asset Prices: A Theory of Market Equilibrium Under 
Conditions of Risk." \textit{J. Finance} 19 (September 1964): 425-42.

\hangindent 5mm
\noindent ---------. \textit{Portfolio Theory and Capital Markets}: New York: McGraw-Hill, 1970.

\hangindent 5mm
\noindent Sprenkle, Case. ``Warrant Prices as Indications of Expectations." \textit{Yale Econ. Essays} 1 
(1961):  179-232.  Reprinted in Cootner (1967), 412-74.

\hangindent 5mm
\noindent Stoll, Hans R. ``The Relationship Between Put and Call Option Prices." \textit{J. Finance} 24 
(December 1969): 802-24.

\hangindent 5mm
\noindent Thorp, Edward O., and Kassouf, Sheen T. \textit{Beat the Market.} New York: Random House, 1967.

\hangindent 5mm
\noindent Treynor, Jack L. ``Implications for the Theory of Finance." Unpublished memorandum, 
1961. (a)

\hangindent 5mm
\noindent ---------. ``Toward a Theory of Market Value of Risky Assets." Unpublished memorandum, 
1961. (b)

\endgroup 

\begin{thebibliography}{9}
\bibitem{Ayres1963}
Ayres, Herbert F. 
``Risk Aversion in the Warrants Market." 
\textit{Indus. Management Rev.} 4 
(Fall 1963): 497-505.  Reprinted in Cootner (1967), pp. 497-505.
\end{thebibliography}

\end{document}